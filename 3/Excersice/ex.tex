\documentclass[12pt]{article}
\renewcommand{\baselinestretch}{1.4}
\usepackage{amsthm,amssymb,amsmath,graphicx}
\usepackage{color}
\usepackage{url}
\usepackage[top=2.5cm, bottom=2.5cm, left=2cm, right=2cm]{geometry}
\usepackage[pagebackref=false,colorlinks,linkcolor=blue,citecolor=blue]{hyperref}
\usepackage{babel}
\usepackage{blindtext}
\usepackage{graphicx}
\usepackage{subcaption}
\setlength{\parindent}{0pt}
\usepackage[localise=on]{xepersian}
\usepackage{xepersian}
\settextfont{IRMitra}

\makeatletter
\newcommand{\xRightarrow}[2][]{\ext@arrow 0359\Rightarrowfill@{#1}{#2}}
\makeatother

\newcounter{mynumber}
\setcounter{mynumber}{1}
\newcommand{\mynum}{\arabic{mynumber}\stepcounter{mynumber}}
\newenvironment{ques}[0]{\textbf{سوال \mynum.}}{}
\newtheorem{قضیه}{قضیه}

\title{حل تمرین آرایه و پشته}
\author{
	ساختمان داده ها و الگوریتم
}
\date{بهار ۱۴۰۰}

\begin{document}
	\maketitle
	\setRTL 

	\begin{ques}
		\textbf{(ارشد ۹۷)}
		در یک آرایه عددی $A$
		به طول 
		$m$ 
		برخی از عناصر ۱ تا 
		$n$ 
		که $m<n$ 
		است، ظاهر شده‌اند. بهترین الگوریتم برای
		یافتن بزرگ‌ترین عنصری که در آرایه ظاهر نشده است؛ دارای چه مرتبه زمانی است؟
		\\۱)
		$O(n)$
		\\۲)
		$O(m)$
		\\۳)
		$O(n\log n)$
		\\۴)
		$O(m\log m)$
		\\
	\end{ques}
	\quad\\
	\begin{ques}
		\textbf{(ارشد ۹۶)}
		می‌خواهيم در یک آرايه 
		$n$تایی
		تایی نامرتب، عنصری که حداقل 
		$\lfloor\frac{n}{2}\rfloor+1$
		بار تکرار شده است را بيابيم؛ کدام گزینه
صحیح است؟
\\۱) الگوریتمی با هزینه حداکثر $O(n)$ وجود دارد
\\۲) الگوریتمی با هزینه حداکثر $O(\sqrt{n})$ وجود دارد
\\۳) بهترین الگوریتم با زمان اجرای میانگین $O(n)$ و مبتنی بر درهم‌سازی است.
\\۴) بهترین الگوریتم با زمان اجرای میانگین $O(n\log n)$ و مبتنی بر مرتب‌سازی است.
\\
	\end{ques}
	\quad\\
	\begin{ques}
		\textbf{(ارشد ۹۵)}
		میخواهیم برای عدد صحیح و مثبت $n$
		بزرگ‌ترین عدد صحیح و مثبت $x$
		را پیدا کنیم که $x^2\leq n$
		باشد. بهترین الگوریتم برای یافتن چنین عددی از چه مرتبه زمانی است؟
		\\۱)
		$O(n)$
		\\۲)
		$O(\sqrt{n})$
		\\۳)
		$O(\log n)$
		\\۴)
		$O(n\log n)$
	\end{ques}
	\newpage
	\begin{ques}
		\textbf{(ارشد ۹۵)}
		اگر الگوریتم جستجوی دودویی را برای جستجوی عناصر آرایه
		زیر
		به کار ببریم, میانگین تعداد مقایسه‌ها برای جستجوی موفق و ناموفق، به طور تقریبی چقدر است؟
		$$A[1..9]=[-1,2,10,20,25,29,35,45,50]$$
		۱) موفق ۲,۸ ناموفق ۳,۸
		\\۲) موفق ۲,۸ ناموفق ۲,۸ 
		\\۳) موفق ۲,۸ ناموفق ۳,۴ 
		\\۴) موفق ۳,۴ ناموفق ۳,۴
	\\
	\end{ques}
	\\
	\begin{ques}
		\textbf{(ارشد ۸۵)}
		در عبارت محاسباتی زیر، عملگر 
		$+$
		به عملگر 
		$\times$
		الوییت دارد، این عبارت معادل کدام عبارت پیشوندی زیر است؟
		$$((2+3)\times 4+5\times(6+7)\times 8 )+9$$
		۱) $+\times++234\times\times 5 + 6789$
		\\۲) $+\times\times\times+23+45+6789$
		\\۳) $++\times+234\times\times 5+6789$
		\\۴) $\times\times\times++23+45+6789$
	\\
	\end{ques}
	\\
	\begin{ques}
		\textbf{(ارشد ۹۸)}
		اعداد ۱ تا ۶ به ترتیب صعودی در ورودی یک پشته داده شده است.
		کدام یک از موارد زیر را نمی‌توان با هیچ ترتیبی از درج و حذف
		در خروجی داشته باشیم؟(اعداد را از چپ به راست بخوانید)
		\\۱) ۱۲۳۵۶۴
		\\۲) ۲۱۵۳۴۶
		\\۳) ۳۲۴۶۵۱
		\\۴) ۴۳۲۱۶۵
		\\
	\end{ques}
	\\
	\begin{ques}
		\textbf{(ارشد ۹۸)}
		فرض کنید k پشته $S_1,\cdots,S_k$ را در اختیار داریم, تنها اعمال مجاز گرفتن یک ورودی و درج
 کردن آن داخل پشته $S_k$ و قرار دادن در خروجی، حذف کردن یک عنصر از پشته $S_i$ (برای 
$i<k$) و درج کردن آن داخل پشته $S_{i+1}$ است. فرض کنید اعداد $1,\cdots,n$ به ترتیب 
کوچک به بزرگ به ‌عنوان ورودی داده می‌شود؛ کوچکترین $k$ که می‌توان همه‌ی جایگشت‌های 
$1,\cdots,n$  را با اعمال مجاز گفته‌شده تولید کرد، در بین گزینه‌ها کدام است؟
	\\۱) $1$
	\\۲) $2$
	\\۳) $n$
	\\۴) $n-1$
	\end{ques}
\end{document}
\documentclass[12pt]{article}
\renewcommand{\baselinestretch}{1.4}
\usepackage{amsthm,amssymb,amsmath,graphicx}
\usepackage{color}
\usepackage{url}
\usepackage[top=2cm, bottom=2.5cm, left=2cm, right=2cm]{geometry}
\usepackage[pagebackref=false,colorlinks,linkcolor=blue,citecolor=blue]{hyperref}
\usepackage{babel}
\usepackage{blindtext}
\usepackage{graphicx}
\usepackage{subcaption}
\setlength{\parindent}{0pt}
\usepackage[localise=on]{xepersian}
\usepackage{xepersian}
\settextfont{IRMitra}

\makeatletter
\newcommand{\xRightarrow}[2][]{\ext@arrow 0359\Rightarrowfill@{#1}{#2}}
\makeatother

\newcounter{mynumber}
\setcounter{mynumber}{1}
\newcommand{\mynum}{\arabic{mynumber}\stepcounter{mynumber}}
\newenvironment{ques}[0]{\textbf{سوال \mynum.}}{}
\newtheorem{قضیه}{قضیه}

\title{حل تمرین صف و لیست پیوندی}
\author{
	ساختمان داده ها و الگوریتم
}
\date{بهار ۱۴۰۰}

\begin{document}
	\maketitle
	\setRTL 
	\begin{ques}
		\textbf{(ارشد ۹۴)}
		فرض کنید صف $Q$ با یک آرایه‌ی حلقوی به اندازه‌ی $m$ پیاده‌سازی شده است که اندیس‌های آن از صفر تا $m-1$ است و عناصر آن به صورت چرخه‌ای و در جهت ساعتگرد ذخیره شده‌اند. مولفه‌های $front(Q)$ و $rear(Q)$ به ترتیب اندیس اولین عنصر و عنصر بعد از آخرین عنصر صفر را ذخیره می‌کنند. تعداد عناصر داخل صف و شرط پر بودن صف کدام گزینه زیر است؟ 
\\۱) $rear(Q)-front(Q)+1 \mod m$ \quad و \quad $front(Q)=rear(Q)$
\\۲) $rear(Q)-front(Q) \mod m$ \quad و \quad $front(Q)=rear(Q)$
\\۳) $rear(Q)-front(Q)+1 \mod m$ \quad و \quad $front(Q)=rear(Q)+1 \mod m$
\\۴) $rear(Q)-front(Q) \mod m$ \quad و \quad $front(Q)=rear(Q)+1 \mod m$
\\
	\end{ques}
	\\
	\begin{ques}
		\textbf{(ارشد ۹۳)}
		برای ساخت یک صف $Q$ از دو پشته $S_1$ و $S_2$ استفاده می‌کنیم. برای درج $x$ در انتهای $Q$ ، عمل $Push(S_1,x)$ را انجام می‌دهیم. برای حذف یک عنصر از ابتدای $Q$ ، اگر $S_2$ خالی نباشد، عمل $Pop(S_2)$ را انجام می‌دهیم. در غیر این صورت، همه‌ی عناصر $S_1$ را به ترتیب $Pop$ کرده و $Push(S_2)$ می‌کنيم. اکنون عمل $Pop$ بر روی $S_2$ عنصر ابتدایی $Q$ را بر می‌گرداند.
\\
اگر بر روی $Q$ که در ابتدا خالی است، ۱۰۰ عمل صورت گیرد (درج در انتها، حذف از ابتدا یا هر ترتیب دل‌خواهی از آن‌ها) حداکثر هزینه چه مقدار خواهد بود؟ فرض کنید هر $Push$ و هر $Pop$ بر روی هر یک از این دو پشته ۱ واحد هزینه دارد.
\\۱) $150$
\\۲) $151$
\\۳) $199$
\\۴) $200$
\\
	\end{ques}
\\
	\begin{ques}
		\textbf{(دکتری ۹۵)}
		داده ساختار صف با سه عملیات افزودن به ابتدای صف، حذف از انتهای صف و استخراج عنصر کمینه را در نظر بگیرید. بهترین پیاده‌سازی
ممکن برای این داده ساختار هر یک از سه عملیات فوق را به‌صورت «سرشکن» در چه زمانی پشتیبانی می‌کند؟ بهترین گزینه را انتخاب کنید.
\\۱) هر سه عملیات $O(1)$
\\۲) هر سه عملیات $O(\log n)$
\\۳) درج و حذف $O(1)$ و استخراج کمینه $O(\log n)$
\\۴) درج و حذف $O(\log n)$ و استخراج کمینه $O(n)$
\\
	\end{ques}
	\\
	\begin{ques}
		\textbf{(ارشد ۹۴)}
		یک لیست پیوندی خطی با دو اشاره‌گر به ابتدا و انتهای آن در نظر بگیرید. کدام یک از اعمال زیر وابسته به طول لیست می‌باشد؟
\\۱) حذف عنصر از ابتدا لیست
\\۲) اضافه‌کردن عنصر به ابتدا لیست
\\۳) اضافه‌کردن عنصر به انتهای لیست
\\۴) حذف عنصر از انتهای لیست
\\
	\end{ques}
	\\
	\begin{ques}
		\textbf{(ارشد ۹۳)}
		روی لیست پیوندی و دوسویه‌ی $Q$ که عناصر آن عدد هستند و اشاره‌گر به عنصر اول و آخر آن را داریم، اعمال زیر تعریف شده‌اند:
\\$Delete(k)$ :  $k$ عنصر ابتدای $Q$ را به ترتیب حذف می‌کند.
\\$Append(c)$ : عنصر آخر $Q$ را نگاه می‌کند، اگر مقدارش از $C$ بیشتر بود آن را حذف می‌کند. این کار را تکرار می‌کند تا عنصر انتهایی کمتر یا مساوی $C$ شود (يا $Q$ تهی شود). در آن صورت عنصر $C$ را به انتههای صف درج می‌کند.
\\اگر دنباله‌ای از n تا از این دو عمل را با ترتیب دلخواه روی یک لیست تهی $Q$ انجام دهیم. مجموع کل هزینه‌ها به کدام گزینه زیر نزدیک‌تر است؟
\\۱) $n-k$
\\۲) $n$
\\۳) $2n$
\\۴) $3n$
	\end{ques}

\end{document}